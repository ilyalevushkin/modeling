\documentclass[a4paper,14pt]{extarticle}

\usepackage[T2A]{fontenc}			
\usepackage[utf8]{inputenc}			
\usepackage[english,russian]{babel}

\usepackage[
bookmarks=true, colorlinks=true, unicode=true,
urlcolor=black,linkcolor=black, anchorcolor=black,
citecolor=black, menucolor=black, filecolor=black,
]{hyperref}

\usepackage{color}
\usepackage{caption}
\DeclareCaptionFont{white}{\color{black}}
\DeclareCaptionFormat{listing}{\colorbox{white}{\parbox{\textwidth}{#1#2#3}}}
\captionsetup[lstlisting]{format=listing,labelfont=white,textfont=white}

\usepackage{amsmath,amsfonts,amssymb,amsthm,mathtools} 
\usepackage{wasysym}

\usepackage{graphicx}
%\usepackage[cache=false]{minted}
\usepackage{cmap}
\usepackage{indentfirst}

\usepackage{longtable}

\usepackage{listings} 
\usepackage{fancyvrb}

\usepackage{geometry}
\geometry{left=2cm}
\geometry{right=1.5cm}
\geometry{top=1cm}
\geometry{bottom=2cm}

\setlength{\parindent}{5ex}
\setlength{\parskip}{0.5em}

\usepackage{color}
\usepackage[cache=false, newfloat]{minted}
\newenvironment{code}{\captionsetup{type=listing}}{}
\SetupFloatingEnvironment{listing}{name=Листинг}
 
 
 \begin{document}
 	
 	\[v_l \in V, l = 1, n_v\]
 	
 	\[h_k \in H, k = 1,n_h\]
 	
 	\[y_j \in Y, j = 1, n_k\]
 	
 	$\vec{x(t)} = (x_1(t), x_2(t), ..., x_{n_x}(t))$
 	
 	$\vec{v(t)} = (v_1(t), ..., v_{n_v}(t))$

	$\vec{h(t)} = (h_1(t), ..., h_{n_h}(t))$
	
	$\vec{y(t)} = (y_1(t), ..., y_{n_y}(t))$
 	
 	$\vec{y(t)} = F_s(\vec{x}, \vec{v}, \vec{h}, t)$
 	
 	$A_s$
 	
 	$\vec{z(t)} = (z_1(t), ..., z_{n_z}(t))$
 	
 	$\vec{z(t)} = \phi(\vec{z}, \vec{x}, \vec{v}, \vec{h}, t)$
 	
 	$\vec{y(t)} = F(\vec{z}, t)$
 	
 	$\vec{y(t)} = F(\phi(\vec{z}, \vec{x}, \vec{v}, \vec{h}, t))$
 	
 	\begin{center}
 		\begin{longtable}[h!]{|p{0.3\linewidth}|p{0.4\linewidth}|p{ 0.3\linewidth}|}
 			\hline
 			{Обозначение} & {Процесс функционирования} & {типовая математическая схема}\\
 			\hline
 			{D} & {Непрерывно-детерменированный подход} & {Дифференциальные уравнения} \\
 			\hline
 			{F} & {Дискретно-детерменированный} & {Конечные автоматы}\\
 			\hline
 			{P} & {Дискретно-стохастический} & {Вероятностный автомат}\\
 			\hline
 			{Q} & {Непрерывно-стохастический} & {Система массового обслуживания}\\
 			\hline
 			{A} & {Универсальный процесс} & {Агрегативные системы}\\
 			\hline
 			\end{longtable}
 		\end{center}
 	
 	\[
 	L\frac{d^2q(t)}{dt^2} + \frac{q(t)}{C_k} = 0
 	\]
 	
 	
 	\[
 	M_n l_n^2 \frac{d^2 Q(t)}{dt^2} + M_m g l_m Q(t) = 0
 	\]
 	
 	Уравнения похожи.
 	
 	$h_2 \frac{d^2 z(t)}{dt^2} + h_1 \frac{dz(t)}{dt} + h_0 z(t) = 0$ - Независимо от природы объекта мы записываем содержательно...
 	
 	
 	
\end{document}