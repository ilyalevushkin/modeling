\documentclass[a4paper,12pt]{article}

\usepackage[T2A]{fontenc}			
\usepackage[utf8]{inputenc}			
\usepackage[english,russian]{babel}	

\usepackage[
bookmarks=true, colorlinks=true, unicode=true,
urlcolor=black,linkcolor=black, anchorcolor=black,
citecolor=black, menucolor=black, filecolor=black,
]{hyperref}

\usepackage{color}
\usepackage{caption}
\DeclareCaptionFont{white}{\color{black}}
\DeclareCaptionFormat{listing}{\colorbox{white}{\parbox{\textwidth}{#1#2#3}}}
\captionsetup[lstlisting]{format=listing,labelfont=white,textfont=white}

\usepackage{amsmath,amsfonts,amssymb,amsthm,mathtools} 
\usepackage{wasysym}

\usepackage{graphicx}
%\usepackage[cache=false]{minted}
\usepackage{cmap}
\usepackage{indentfirst}

\usepackage{listings} 
\usepackage{fancyvrb}

\usepackage{geometry}
\geometry{left=2cm}
\geometry{right=1.5cm}
\geometry{top=1cm}
\geometry{bottom=2cm}

\setlength{\parindent}{5ex}
\setlength{\parskip}{0.5em}

\usepackage{pgfplots}
\usetikzlibrary{datavisualization}
\usetikzlibrary{datavisualization.formats.functions}

\begin{document}
	\lstset{ %
		language=C,                 % выбор языка для подсветки (здесь это С)
		basicstyle=\small\sffamily, % размер и начертание шрифта для подсветки кода
		numbers=left,               % где поставить нумерацию строк (слева\справа)
		numberstyle=\tiny,           % размер шрифта для номеров строк
		stepnumber=1,                   % размер шага между двумя номерами строк
		numbersep=5pt,                % как далеко отстоят номера строк от подсвечиваемого кода
		backgroundcolor=\color{white}, % цвет фона подсветки - используем \usepackage{color}
		showspaces=false,            % показывать или нет пробелы специальными отступами
		showstringspaces=false,      % показывать или нет пробелы в строках
		showtabs=false,             % показывать или нет табуляцию в строках
		frame=single,              % рисовать рамку вокруг кода
		tabsize=2,                 % размер табуляции по умолчанию равен 2 пробелам
		captionpos=t,              % позиция заголовка вверху [t] или внизу [b] 
		breaklines=true,           % автоматически переносить строки (да\нет)
		breakatwhitespace=false, % переносить строки только если есть пробел
		escapeinside={\%*}{*)}   % если нужно добавить комментарии в коде
	}

\begin{figure}[h!]
	\begin{center}
		{\includegraphics[width = \textwidth]{titul.jpg}}
	\end{center}
\end{figure}

%\vspace*{10mm}

\huge
\begin{center}
	Лабораторная работа №1
\end{center}

\vspace*{10mm}

\large
\begin{center}
	Тема: <<Приближенный аналитический метод Пикара, сравнение
	с численными методами>>
\end{center}

\vspace*{20mm}

\large
\begin{flushleft}
	Студент: Левушкин И.К. \\
	Группа: ИУ7-62Б \\
	Оценка (баллы): \\
	Преподаватель: Градов В.М.
\end{flushleft}

\vspace*{50mm}

\large
\begin{center}
	Москва, 2020 г.
\end{center}

\thispagestyle{empty}

\newpage

%\section*{Условие задачи}

Рассмотрим задачу с начальным условием для дифференциального
уравнения (задачу Коши):
	
\[
\left\{
\begin{aligned}
y'(x) &= f(x, y) \\
y(\xi) &= \eta \\
\end{aligned}
\right.
\]

Решение можно найти приближенным аналитическим методом Пикара:

\[
\begin{aligned}
y^{(n+1)}(x) &= \eta + \int\limits_{\xi}^x f(t, y^{(n)}(t)) dty \\
y^{(0)}(x) &= \eta \\
\end{aligned}
\]

Покажем на конкретном примере:

\[
\begin{aligned}
y'(x) &= x^2 + y^2 \\
y(0) &= 0 \\
y^{(1)}(x) &= 0 + \int\limits_{0}^x t^2 dty = \frac{t^3}{3} \\
y^{(2)}(x) &= 0 + \int\limits_{0}^x \bigg[t^2 + \bigg(\frac{t^3}{3}\bigg)^2\bigg] dty = 
\frac{t^3}{3} + \frac{t^7}{7 \cdot 9} =  \frac{t^3}{3} \bigg[1 + \frac{t^4}{21}\bigg]\\
y^{(3)}(x) &= 0 + \int\limits_{0}^x \bigg[t^2 + \bigg(
\frac{t^3}{3} \bigg[1 + \frac{t^4}{21}\bigg]\bigg)^2\bigg] dt = 
\frac{t^3}{3} + \frac{t^7}{7 \cdot 9} =  \frac{t^3}{3} \bigg[1 + \frac{t^4}{21}\bigg] =\\
&= \frac{x^3}{3} \bigg[1 + \frac{x^4}{21} +
\frac{2x^8}{693} + \frac{x^{12}}{19845}\bigg] \\
\dots
\end{aligned}
\]

Чем больше итераций $n$, тем точнее результат.

Кроме того, эту задачу можно решить численными методами. Следующая формула
для явного способа:

\[
\begin{aligned}
y_{n+1} &= y_n + h \cdot f(x_n, y_n) \\
\end{aligned}
\]

Похожим образом выглядит неявный метод:

\[
\begin{aligned}
y_{n+1} &= y_n + h \cdot f(x_{n+1}, y_{n+1}) \\
\end{aligned}
\]

%Здесь $f(x_n, y_n) = x_n^2 + y_n^2$.

Чем меньше шаг $h$, тем точнее значение получим.

Стоит заметить, что для всех рассмотренных методов результат будет тем лучше,
чем ближе значение $x$ к $\xi$.

Реализованная программа производит расчет для $f(x, y) = x^2 + y^2$.
Исходя из этого, можно упростить нахождения решения в неявном виде:
получаем квадратное уравнение и в качестве решения берем меньший
корень.


\end{document}

